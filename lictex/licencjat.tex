\documentclass[a4paper,12pt]{book}


\usepackage[T1,plmath]{polski}
\usepackage[utf8]{inputenc} % w razie kłopotów spróbować: \usepackage[utf8x]{inputenc}
\usepackage{fancyhdr} % nagłówki i stopki
\usepackage{indentfirst} % WAŻNE, MA BYĆ!
\usepackage[pdftex]{graphicx} % to do wstawiania rysunków
\usepackage{amsfonts} % pakiety od AMS, ułatwiają składanie pewnych techniczno-matematcyznych rzeczy
\usepackage{amsmath} % to do dodatkowych symboli, przydatne
\usepackage{amssymb} % to też do dodatkowych symboli, też przydatne
\usepackage{amsthm}
\usepackage[pdftex,
            left=1.1in,right=1.1in,
            top=1.1in,bottom=1.1in]{geometry} % marginesy
\usepackage{float}
\usepackage[font=small,labelfont=bf]{caption}
\usepackage{cite}

\usepackage[colorlinks=true]{hyperref} % odnośniki interaktywne w PDFie
\hypersetup{allcolors=blue} % odnośniki proszę zostawić kolorowe -- a jeśli ktoś nie lubi to zmienić na czarne dopiero w ostatniej wersji!
\usepackage{listings}
\lstset{
    basicstyle=\footnotesize\tt,
    numbers=left,
    numberstyle=\tiny,
    frame=tb,
    tabsize=4,
    columns=fixed,
    showstringspaces=false,
    showtabs=false,
    keepspaces,
    captionpos=t,    
    commentstyle=\color{red},
    keywordstyle=\color{blue}
}
\newfloat{lstfloat}{htbp}{lolst}[chapter]
\floatname{lstfloat}{Listing}
\def\lstfloatautorefname{Listing}

% jesli potrzeba, można oczywiście wstawić inne pakiety i swoje definicje...



% definicje nagłówków i stopek
\pagestyle{fancy}
\renewcommand{\chaptermark}[1]{\markboth{#1}{}}
\renewcommand{\sectionmark}[1]{\markright{\thesection\ #1}}
\fancyhf{}
\fancyhead[LE,RO]{\footnotesize\bfseries\thepage}
\fancyhead[LO]{\footnotesize\rightmark}
\fancyhead[RE]{\footnotesize\leftmark}
\renewcommand{\headrulewidth}{0.5pt}
\renewcommand{\footrulewidth}{0pt}
\addtolength{\headheight}{1.5pt}
\fancypagestyle{plain}{\fancyhead{}\cfoot{\footnotesize\bfseries\thepage}\renewcommand{\headrulewidth}{0pt}}


% interlinia
\linespread{1.25}

\bibliographystyle{plain}

\begin{document}
\begin{titlepage}

\begin{tabular}{c@{\hspace{21mm}}|@{\hspace{5mm}}l}
\vspace{-20mm} & \\
\multicolumn{2}{l}{\hspace{-12.5mm} \includegraphics[width=8cm]{LogoUMCS.jpg}} \\
\multicolumn{2}{@{\hspace{20mm}}l}{\vspace{-4mm}} \\
\multicolumn{2}{@{\hspace{28mm}}l}{\Large \sf UNIWERSYTET MARII
	CURIE-SKŁODOWSKIEJ} \\
\multicolumn{2}{@{\hspace{28mm}}l}{\vspace{-4mm}} \\
\multicolumn{2}{@{\hspace{28mm}}l}{\Large \sf W LUBLINIE} \\
\multicolumn{2}{@{\hspace{28mm}}l}{\vspace{-4mm}} \\
\multicolumn{2}{@{\hspace{28mm}}l}{\Large \sf Wydział Matematyki, Fizyki i
	Informatyki} \\
\multicolumn{2}{@{\hspace{28mm}}l}{\vspace{21mm}} \\
& {\sf Kierunek: \textbf{informatyka} } \\
%& {\sf Specjalność: \textbf{informatyczna}} \\ % wpisujemy tylko jeśli jest!!!
& \\\\\\
& {\sf \large \bfseries Mateusz Kowal} \\
& {\sf nr albumu: 318683} \\
& \\\\\\
& \Large \sf \bfseries Środowisko programowalnego \\
& \Large \sf \bfseries  bezzałogowego statku powietrznego \\
& \Large \sf \bfseries z~użyciem mikrokomputera Raspberry~Pi \\
& {\large \sf A programmable unmanned aerial vehicle environment} \\
& {\large \sf based on a Raspberry Pi microcomputer} \\
& \\
& \\
& \\
& {\sf Praca licencjacka}  \\
& \vspace{-7mm} \\
&  {\sf napisana w Katedrze Oprogramowania Systemów Informatycznych} \\
&  {\sf Instytutu Informatyki i Matematyki UMCS} \\
& \vspace{-7mm} \\
& {\sf pod kierunkiem \bfseries dr hab. profesor uczelni Beaty Byliny} \\
\multicolumn{2}{@{\hspace{28mm}}l}{\vspace{15mm}} \\
\multicolumn{2}{@{\hspace{28mm}}l}{\textbf{\textsf{Lublin 2026}}}
\end{tabular}
\end{titlepage}


\sloppy

\thispagestyle{empty}

\newpage{}

\thispagestyle{empty}

\newpage{}

\tableofcontents{}

\chapter*{Wstęp}
\addcontentsline{toc}{chapter}{Wstęp} % ...ale w spisie treści ma być...
\chaptermark{Wstęp} % ...i w paginie górnej

Tu treść wstępu (który piszemy na końcu, po napisaniu całej pracy).

\chapter{Wprowadzenie do bezzałogowych statków powietrznych}

\section{Podstawowe pojęcia}

Bezzałogowy statek powietrzny, skrót BSP, to kategoria pojazdów poruszających się w powietrzu 
do której zaliczane są samoloty, śmigłowce i pojazdy wielowirnikowe (w tym drony), które 
odbywają loty bez pilota na pokładzie i bez możliwości zabierania pasażerów. Sterowanie może 
się odbywać zdalnie przez pilota naziemnego lub przez komputer, jeśli pojazd jest autonomiczny.\cite{web:easa:uavdef}

Komputer pokładowy jest urządzeniem niezbędnym do działania BSP. Jego głównym zadaniem jest sterowanie silnikami pojazdu 
oraz otrzymywanie i przetwarzanie sygnałów docierających z aparatury sterującej. W tym przypadku rolę komputera pokładowego pełni Pixhawk 6C.

Komputer asystujący to komputer znajdujący się na pokładzie BSP, połączony z komputerem pokładowym. Jego rolą jest wykonywanie zadań, które są zbyt 
skomplikowane obliczeniowo by móc powierzyć je komputerowi pokładowemu lub wymagają elementów przez komputer pokładowy nieposiadanych i niewspieranych. 
Komputerem asystującym w przypadku tego projektu jest Raspberry Pi 5B.

\section{Zadania i ograniczenia komputera pokładowego}
        \subsection{Podstawowe zadania komputera pokładowego}
        Rolą każdego komputera pokładowego jest przede wszystkim sterowanie silnikami pojazdu którego jest częścią. Komputer pokładowy musi więc posiadać odpowiednie 
        interfejsy do komunikacji z silnikami oraz interfejs do otrzymywania poleceń. W przypadku komputerów pokładowych używanych w BSP wielowirnikowych niezbędny jest również żyroskop, 
        dzięki któremu komputer pokładowy zna swoją pozycję i jest w stanie utrzymać pojazd poziomo lub odpowiednio odchylić go zgodnie z poleceniami które otrzyma.
        \subsection{Opcjonalne zastosowania autopilota}
        W zależności od modelu komputera pokładowego oraz zainstalowanego na nim oprogramowania może on dodatkowo obsługiwać wiele urządzeń peryferyjnych i posługiwać się bardziej złożoną logiką.
        Przeciętny komputer pokładowy dostępny na rynku cywilnym oferuje możliwość podłączenia zewnętrznego modułu GPS czy kamery, a także samodzielnie przechowywać i wykonywać polecenia 
        pozwalające na lot autonomiczny. Standardem jest również możliwość ustawienia procedury bezpieczeństwa (ang. failsafe) która decyduje jak zachowa się pojazd w razie utraty podłączenia
        z kontrolerem naziemnym lub jakimkolwiek innym urządzeniem wydającym polecenia komputerowi pokładowemu.
    \section{Rozszerzenie możliwości BSP dzięki komputerom asystującym}
    Jeśli użytkownik chce wykorzystać swój BSP w bardziej nietypowy sposób, mogą stanąć mu na drodze ograniczenia sprzętowe. Komputery pokładowe oferują zazwyczaj niewielką moc obliczeniową
    i ograniczoną ilość i różnorodność gniazd do podłączenia urządzeń peryferyjnych. Jednym ze sposobów przystosowania BSP do naszych wymagań jest użycie komputera asystującego który 
    rozszerzy możliwości jednostki. Przykładowo, użytkownik może chcieć użyć swojego BSP do monitorowania dużego obszaru. W tym celu chciałby zamontować kamery z wielu stron, jednak 
    użyty przez niego komputer pokładowy ma możliwość podpięcia tylko jednej kamery. Użycie komputera asystującego może być wtedy rozwiązaniem problemu. W zależności od użytego komputera 
    asystującego można użyć nawet tak złożonych obliczeniowo programów jak sztuczna sieć neuronowa analizująca dane z urządzeń wejściowych i wydająca polecenia jednostce.
\addcontentsline{toc}{chapter}{Bibliografia} % też ręczne dodanie do spisu treści, jak Wstęp
\label{sec:bibliografia}
\bibliography{MKowalBibliografia}

\end{document}
