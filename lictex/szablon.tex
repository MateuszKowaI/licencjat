\documentclass[a4paper,12pt]{book} % nie: report!


\usepackage[T1,plmath]{polski} % lepiej to zamiast babel!
\usepackage[utf8]{inputenc} % w razie kłopotów spróbować: \usepackage[utf8x]{inputenc}
\usepackage{fancyhdr} % nagłówki i stopki
\usepackage{indentfirst} % WAŻNE, MA BYĆ!
\usepackage[pdftex]{graphicx} % to do wstawiania rysunków
\usepackage{amsfonts} % pakiety od AMS, ułatwiają składanie pewnych techniczno-matematcyznych rzeczy
\usepackage{amsmath} % to do dodatkowych symboli, przydatne
\usepackage{amssymb} % to też do dodatkowych symboli, też przydatne
\usepackage{amsthm}
\usepackage[pdftex,
            left=1.1in,right=1.1in,
            top=1.1in,bottom=1.1in]{geometry} % marginesy
\usepackage{float}
\usepackage[font=small,labelfont=bf]{caption}
\usepackage{cite}

\usepackage[colorlinks=true]{hyperref} % odnośniki interaktywne w PDFie
\hypersetup{allcolors=blue} % odnośniki proszę zostawić kolorowe -- a jeśli ktoś nie lubi to zmienić na czarne dopiero w ostatniej wersji!
\usepackage{listings}
\lstset{
    basicstyle=\footnotesize\tt,
    numbers=left,
    numberstyle=\tiny,
    frame=tb,
    tabsize=4,
    columns=fixed,
    showstringspaces=false,
    showtabs=false,
    keepspaces,
    captionpos=t,    
    commentstyle=\color{red},
    keywordstyle=\color{blue}
}
\newfloat{lstfloat}{htbp}{lolst}[chapter]
\floatname{lstfloat}{Listing}
\def\lstfloatautorefname{Listing}

% jesli potrzeba, można oczywiście wstawić inne pakiety i swoje definicje...



% definicje nagłówków i stopek
\pagestyle{fancy}
\renewcommand{\chaptermark}[1]{\markboth{#1}{}}
\renewcommand{\sectionmark}[1]{\markright{\thesection\ #1}}
\fancyhf{}
\fancyhead[LE,RO]{\footnotesize\bfseries\thepage}
\fancyhead[LO]{\footnotesize\rightmark}
\fancyhead[RE]{\footnotesize\leftmark}
\renewcommand{\headrulewidth}{0.5pt}
\renewcommand{\footrulewidth}{0pt}
\addtolength{\headheight}{1.5pt}
\fancypagestyle{plain}{\fancyhead{}\cfoot{\footnotesize\bfseries\thepage}\renewcommand{\headrulewidth}{0pt}}


% interlinia
\linespread{1.25}

\bibliographystyle{plain}

\begin{document}
\begin{titlepage}

\begin{tabular}{c@{\hspace{21mm}}|@{\hspace{5mm}}l}
\vspace{-20mm} & \\
\multicolumn{2}{l}{\hspace{-12.5mm} \includegraphics[width=8cm]{LogoUMCS.jpg}} \\
\multicolumn{2}{@{\hspace{20mm}}l}{\vspace{-4mm}} \\
\multicolumn{2}{@{\hspace{28mm}}l}{\Large \sf UNIWERSYTET MARII
	CURIE-SKŁODOWSKIEJ} \\
\multicolumn{2}{@{\hspace{28mm}}l}{\vspace{-4mm}} \\
\multicolumn{2}{@{\hspace{28mm}}l}{\Large \sf W LUBLINIE} \\
\multicolumn{2}{@{\hspace{28mm}}l}{\vspace{-4mm}} \\
\multicolumn{2}{@{\hspace{28mm}}l}{\Large \sf Wydział Matematyki, Fizyki i
	Informatyki} \\
\multicolumn{2}{@{\hspace{28mm}}l}{\vspace{21mm}} \\
& {\sf Kierunek: \textbf{informatyka/matematyka/geoinformatyka/...} } \\
%& {\sf Specjalność: \textbf{informatyczna}} \\ % wpisujemy tylko jeśli jest!!!
& \\\\\\
& {\sf \large \bfseries Imię Nazwisko} \\
& {\sf nr albumu: \dots} \\
& \\\\\\
& \Large \sf \bfseries Tytuł po polsku, który zwykle\\
& \Large \sf \bfseries jest długi na wiele linijek\\\\[-10pt]
& {\large \sf Title in English} \\
& {\large \sf (also a long one)} \\
& \\
& \\
& \\
& {\sf Praca magisterska/licencjacka/inżynierska}  \\
& \vspace{-7mm} \\
&  {\sf napisana w Katedrze ...} \\
&  {\sf Instytutu Informatyki i Matematyki UMCS} \\
& \vspace{-7mm} \\
& {\sf pod kierunkiem \bfseries stopień/tytuł imię i nazwisko (odmienione!)} \\
\multicolumn{2}{@{\hspace{28mm}}l}{\vspace{15mm}} \\
\multicolumn{2}{@{\hspace{28mm}}l}{\textbf{\textsf{Lublin 2022}}}
\end{tabular}
\end{titlepage}





\sloppy



\thispagestyle{empty}


\newpage{}

\thispagestyle{empty}

\newpage{}



\tableofcontents{}

\chapter*{Wstęp} % z gwiazdką, więc bez numerka...
\addcontentsline{toc}{chapter}{Wstęp} % ...ale w spisie treści ma być...
\chaptermark{Wstęp} % ...i w paginie górnej

Tu treść wstępu (który piszemy na końcu, po napisaniu całej pracy).

\chapter{Podstawowe uwagi typograficzne}

\section{Znaki specjalne}

Kilka słow o~znakach specjalnych poniżej --- przy okazji mamy tu też przykład wyliczenia:
\begin{itemize}
\item
Polskie cudzysłowy piszemy ,,tak'' (za pomocą dwóch przecinków dla otwarcia i~dwóch prawych apostrofów do zamknięcia).
\item
Cudzysłowy pojedyncze `xyz'
\item Inne cudzysłowy (jeśli potrzeba):  <<xyz>>  >>xyz<<
\item
Kreski poziome sprawiają często problemy, więc pamiętamy: myślnik (pauza) rozdziela od siebie myśli --- tak jak tu --- i~pisze się go za pomocą trzech minusów otoczonych spacjami. Jeden minus to łącznik (pisany bez spacji), tak jak tu: biało-czerwony. Półpauzy (pisanej dwoma minusami bez spacji) używamy właściwie tylko do określania przedziałów typu: 7:00--9:00 (czyli od siódmej do dziewiątej).
\item
Wielokropek robimy instrukcją \verb+\ldots{}+, nie trzema kropkami. Powinien wyglądać tak\ldots{}
\end{itemize}

\section{Kroje pisma}

Cytaty z~plików tekstowych, kodów programów, także komendy i~nazwy elementów języka --- pismem mamszynowym:
\begin{verbatim}
\begin{verbatim}...
\verb+...+

{\tt ...}
\texttt{...}
\end{verbatim}

\section{Inne sprawy}

Tytułów i~podpisów nie kończymy kropką, gdy są zdaniem lub równoważnikiem twierdzącym. Ale kończymy pytajnikiem, wielokropkiem lub wykrzyknikiem, gdy są odpowiednio pytające, zawieszone lub wykrzyknikowe.

A,~i~jeszcze twarda spacja --- oznaczamy ją tyldą w kodzie: to i~to i~to i~to i~jeszcze to i~tamto i~to i~to i~to i~to i~to i~to i~to i~to i~to i~to i~to i~to i~to i~to i~to i~to i~to i~to i~to i~to i~to i~to i~to i~to i~to i~to i~to i~to i~to i~to i~to i~to i~to i~to i~to i~to i~to i~to i~to i~to i~to i~to i~to i~to i~to i~to i~to i~to i~to i~tamto\ldots{}

\chapter{Rozdział o tym jak pisać pracę}

\section{Pierwszy podrozdział, w~którym są zawarte chaotyczne wskazówki z poprzedniej wersji jako przykłady}

W tabeli~\ref{tab:przyk} widzimy przykład tabeli z~nagłówkiem i~odnośnikiem. Tabele tworzymy z~nagłówkiem na górze oraz (najlepiej) opcją \texttt{[t!]}.
Natomiast na rysunku~\ref{rys:przyk} --- widzimy przykład rysunku z nagłówkiem i~odnośnikiem.
Każda wstawka pływająca (rysunek, tabela, algorytm, listing, \ldots{}) musi mieć odnośnik w tekście (jak tutaj)!
Rysunki tworzymy z nagłówkiem pod spodem oraz (najlepiej) opcją \texttt{[b!]}.
Rysunki powinny być w formacie PDF; jeśli to niemożliwe, to PNG (w wysokiej rozdzielczości); a~ostatecznie JPG (jak tu). Jeśli chcemy sterować rozmiarem, to zwykle najwygodniej użyć \texttt{width=...}
Ponadto możemy odwoływać się do bibliografii~\cite{Alsolami:2012,Mihalcea:2006,peyret2012:ch7,knuth:1984,latex2e,lesk:1977,web:lang:stats}. Zwracam uwagę, że niezacytowne pozycje z pliku \verb'.bib' nie są dołączane w bibliografii (patrz strona~\pageref{sec:bibliografia})

Jeśli chodzi o wzory, możemy złożyć je na kilka sposobów, w zależności od potrzeb --- w tekście: $e=\lim_{n\to\infty}\left(1+\frac{1}{n}\right)^n$, wyniesiony do osbnej linii
(warto zwrócić uwagę, że drugi i~trzeci są złożone nieco inaczej niż pierwszy):
\[e=\lim_{n\to\infty}\left(1+\frac{1}{n}\right)^n,\] a także wyniesiony z numerem:
\begin{equation}
e=\lim_{n\to\infty}\left(1+\frac{1}{n}\right)^n.
\label{wzor:e}
\end{equation}
Do tego ostatniego możemy się odwołać:~\eqref{wzor:e}.

No i oczywiście listingi --- listing~\ref{lst:przyk} pokazuje, jak zrobić to w~miarę poprawnie\ldots{}
\lstinline+int a, b, c;+

\begin{figure}[b!]
 \begin{center}
  \includegraphics[width=5cm]{LogoUMCS}
 \end{center}
 \caption{Przykładowy rysunek}\label{rys:przyk}
\end{figure}

\begin{table}[t!]
\begin{center}
\caption{Przykładowa tabela}\label{tab:przyk}
 \begin{tabular}{l|c|r}
    slkdjfslj & sdkskd & s;lkdsdk \\
    \hline
    slkjd & skljdsldj& skljdsjdsldj \\
    sljkdslkjd& woieupowiepoweiwiewp & weoiw eppowie wpo \\
 \end{tabular}
\end{center}
\end{table}

\begin{lstfloat}[t!]
\lstset{language=C++}
\begin{lstlisting}[frame=single, caption={Jakieś dwie linijki w~C++ (z~OpenACC)}, label={lst:przyk}]
tab[0:n] = dem[nRows][nCols]; //?
#pragma acc data copy(tab [0:n], slope [0:n])
\end{lstlisting}
\end{lstfloat}

\clearpage{} % warto zrobić, jeśli chcemy się upewnić, że wstawki (rysunki itp.) nie wyjdą poza sekcję/podrozdział

\section{Taka pusta sekcja (podrozdział)}

Tu prawie nic nie ma. :)

\chapter*{Podsumowanie} % jak we wstępie
\addcontentsline{toc}{chapter}{Podsumowanie}
\chaptermark{Podsumowanie}

Tu treść podsumowania (którą --- oczywiście --- też piszemy na końcu).

 % jeśli są listingi:
\cleardoublepage
\addcontentsline{toc}{chapter}{Spis listingów}
\renewcommand{\lstlistlistingname}{Spis listingów}
\lstlistoflistings{}

 % jeśli są tabele:
\cleardoublepage
\addcontentsline{toc}{chapter}{Spis tabel}
\listoftables{}

 % jeśli są rysunki:
\cleardoublepage
\addcontentsline{toc}{chapter}{Spis rysunków}
\listoffigures{}

\cleardoublepage
\addcontentsline{toc}{chapter}{Bibliografia} % też ręczne dodanie do spisu treści, jak Wstęp
\label{sec:bibliografia}
\bibliography{mojabibliografia}

\end{document}
